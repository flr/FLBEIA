
\section{\texttt{FLBEIA} functions} \label{sec:FLBEIAfun}

\subsection{First level function: \texttt{FLBEIA}} \label{sec:1stlvl}

	\texttt{FLBEIA} function is a multistock, multifleet and seasonal simulation algorithm coded in a generic, flexible and extensible way. It is generic because it can be applied to any case study that fit into the model restrictions. The algorithm is made up by third and fourth level functions specified by the user. In addition of the existing functions new ones can be defined and used if necessary. This is why we define the model as flexible and extensible. 
	
	To determine the simulation, the third- and fourth-level functions must be specified in the main function \texttt{FLBEIA}. For this purpose it has a control argument associated to each second level function. These control arguments are lists which include the name of the functions to be used in the simulations and any extra argument required by those functions that is not already contained in the main arguments. 
	
\noindent \texttt{FLBEIA} function is called as follows:

\begin{center}
  \texttt{FLBEIA(biols, SRs, BDs, fleets, covars, indices, advice, 
        main.ctrl, biols.ctrl, fleets.ctrl, covars.ctrl, obs.ctrl, assess.ctrl, advice.ctrl) }
\end{center}

\noindent Main arguments:
\begin{description}
	\item[\texttt{biols}:] An \texttt{FLBiols} object (list of \texttt{FLBiol} objects). The object must be named and the names must be the same as in the \texttt{SRs} object, the \texttt{BDs} object and the \texttt{catches} slots within \texttt{FLFleetExts} object. For details on \texttt{FLBiol} object see Figure~\ref{fig:FLBiol}. 
	\item[\texttt{SRs}:] A list of \texttt{FLSRsim} objects. This object is a simulation version of the original \texttt{FLSR} object. The object must be named and the names must be the same as in the \texttt{FLBiols} object. For details on \texttt{FLSRsim} object see Figure~\ref{fig:FLSRsim}.
	\item[\texttt{BDs}:] A list of \texttt{FLBDsim} objects. This object is similar to \texttt{FLSRs} object but oriented to simulate	population growth in biomass dynamics populations. The object must be named and the names must coincide with those used in \texttt{FLBiols} object. For details about \texttt{FLBDsim} object see Figure~\ref{fig:FLBDsim}.
	\item[\texttt{fleets}:] An \texttt{FLFleetsExt} object (list of \texttt{FLFleetExt} objects). \texttt{FLFleetExt} object is almost equal to the original \texttt{FLFleet} object but the \texttt{FLCatch} object in \texttt{catch} slot has been replaced by \texttt{FLCatchExt} object. The difference between  \texttt{FLCatch}  and \texttt{FLCatchExt} objects is that \texttt{FLCatchExt} has two extra slots \texttt{alpha} and \texttt{beta} used to store Cobb-Douglas production function parameters, $\alpha$ and $\beta$, \citep{Cobb1928, Clark1990}. $\alpha$ corresponds with the exponent of effort and $\beta$ to the exponent of biomass. The \texttt{FLFleetsExt} object must be named and these names must be consistently used in the rest of the arguments. For details about \texttt{FLFleetExt} object see Figure~\ref{fig:FLFleetExt}.
	\item[\texttt{covars}:] An \texttt{FLQuants} object. This object is not used in the most basic configuration of the algorithm. Its content depends on the third or lower level functions that make use of it.  
	\item[\texttt{indices}:] A list of \texttt{FLIndex} objects. Each element in the list corresponds with one stock. The list must be named and the names must be the same as in the \texttt{FLBiols} object. For details about \texttt{FLIndex} object see Figure~\ref{fig:FLIndex}.
	\item[\texttt{advice}:] A list with three elments (usually: TAC, TAE and quota.share). The class and content of its elements depends on two functions, the function in \texttt{fleet.om} defined to simulate fleets' effort and the function used to produce advice in \texttt{advice.mp}. 
\end{description}
   
   
\noindent Control arguments:
\begin{description}
	\item[\texttt{main.ctrl}:]   Controls the behaviour of the main function, \texttt{FLBEIA}. 
	    For details on \texttt{main.ctrl} object see Table~\ref{tb:A3.table1}. 
	\item[\texttt{biols.ctrl}:]  Controls the behaviour of the second level function \texttt{biols.om}. 
	    For details on \texttt{biols.ctrl} object see Table~\ref{tb:A3.table2}.
	\item[\texttt{fleets.ctrl}:] Controls the behaviour of the second level function \texttt{fleets.om}. 
	    For details on \texttt{fleets.ctrl} object see Table~\ref{tb:A3.table3}.
	\item[\texttt{covars.ctrl}:] Controls the behaviour of the second level function \texttt{covars.om}. 
	    For details on \texttt{covars.ctrl} object see Table~\ref{tb:A3.table4}.
	\item[\texttt{obs.ctrl}:]    Controls the behaviour of the second level function \texttt{observation.mp}. 
	    For details on \texttt{obs.ctrl} object see Table~\ref{tb:A3.table5}.
	\item[\texttt{assess.ctrl}:] Controls the behaviour of the second level function \texttt{assessment.mp}. 
	    For details on \texttt{assess.ctrl} object see Table~\ref{tb:A3.table6}.
	\item[\texttt{advice.ctrl}:] Controls the behaviour of the second level function \texttt{advice.mp}. 
	    For details on \texttt{advice.ctrl} object see Table~\ref{tb:A3.table7}.
\end{description}
